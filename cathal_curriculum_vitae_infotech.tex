%%%%%%%%%%%%%%%%%%%%%%%%%%%%%%%%%%%%%%%%%
% Friggeri Resume/CV
% XeLaTeX Template
% Version 1.0 (5/5/13)
%
% This template has been downloaded from:
% http://www.LaTeXTemplates.com
%
% Original author:
% Adrien Friggeri (adrien@friggeri.net)
% https://github.com/afriggeri/CV
%
% License:
% CC BY-NC-SA 3.0 (http://creativecommons.org/licenses/by-nc-sa/3.0/)
%
% Important notes:
% This template needs to be compiled with XeLaTeX and the bibliography, if used,
% needs to be compiled with biber rather than bibtex.
%
%%%%%%%%%%%%%%%%%%%%%%%%%%%%%%%%%%%%%%%%%

\documentclass[]{friggeri-cv} % Add 'print' as an option into the square bracket to remove colors from this template for printing

\begin{document}

\header{}{Cathal Garvey}{Information Security Enthusiast, Programming Polyglot} % Your name and current job title/field

%----------------------------------------------------------------------------------------
%	SIDEBAR SECTION
%----------------------------------------------------------------------------------------

\begin{aside} % In the aside, each new line forces a line break
\section{contact}
22 McDonagh Road,
Ballyphehane,
Cork
~
+353 (0) 87 6363185
~
\href{mailto:cathalgarvey@cathalgarvey.me}{cathalgarvey@cathalgarvey.me}
\href{http://www.indiebiotech.com}{indiebiotech.com}
\href{https://www.github.com/cathalgarvey}{github.com/cathalgarvey}
~
\section{research}
Lean Biotech R\&D
Science Communication
Computational Biology
Synthetic Biology
Precision Gene Therapy
~
\section{programming}
Python, Lua, Coffeescript, Javascript, CSS3 \& HTML5, Mozilla Rust, Haxe, Java, Cobra (CLI), Haskell, ANSI\,C, \LaTeX
~
\section{interests}
Synthetic Biology, Bioinformatics, Programming, Cryptography, Data Analysis \& Visualisation, Open Data, Open Science, Permaculture, Cycling, Writing, Hiking, Linux \& FLOSS
%----------------------------------------------
\end{aside}

%----------------------------------------------------------------------------------------
%	EDUCATION SECTION
%----------------------------------------------------------------------------------------

\section{Formal Education}

\begin{entrylist}
%------------------------------------------------
\entry
{2003--2007}
{Bachelor of Science, Genetics}
{NUI-Cork / University College Cork}
{Focus on Medical Genetics, Bioethics, Virology, Bioinformatics \& Plant Science.
Penultimate year project on plant immune response and soil potentiation of plant
disease resistance. Final year project on murine pregnancy-specific glycoproteins \&
PCR recovery of same. Graduated 2H1 with consistent honours.
}

\end{entrylist}
%------------------------------------------------

%----------------------------------------------------------------------------------------
%	WORK EXPERIENCE SECTION
%----------------------------------------------------------------------------------------

\section{Professional Experience}

\begin{entrylist}
%------------------------------------------------
\entry
{2012--2014}
{Glowbiotics Ltd}
{Cork, Ireland}
{\emph{Founder, Chief Executive Officer} \\
Two years in a venture-funded startup pioneering lean biotechnology and computational
DNA design for synthetic biology. Wrote a number of in-house DNA software design tools,
designed and used in-house hardware designs to reduce costs and accelerate research.
Collated and compiled protocols and SOPs based on deep literature research that saved
time and money while remaining efficient and reliable. Conceived and designed a series of
ambitious DNA-based products with many novel and positive results, using own software.
Key skills: Hardware, Software and Wetware design, Regulatory compliance, Administration,
Synthetic Biology, Computational Biology
}
%------------------------------------------------
\entry
{2014}
{Open Innovation Partners Ltd}
{Cork, Ireland}
{\emph{Contract Software Engineer} \\
Contracted to produce a business intelligence \& data-visualisation platform for co-authorship
data via the NCBI EUtils API in order to perform due diligence consultations and to discover
and assess potential future academic collaborators. Given a name or a MeSH heading, a
connectivity graph with visual indicators of academic impact could be produced and interactively
explored.
Key skills; EUtils XML API, Dynamic XML to JSON conversion, SQL backend storage, Web Application
development (Flask), High Performance Code, Data Processing/Visualisation (D3), UX Design
}
%------------------------------------------------
\entry
{2013--2014}
{Science Gallery}
{Trinity College, Dublin, Ireland}
{\emph{Exhibit Curator} \\
Co-curated a flagship exhibit exploring the future and cultural role of Synthetic
Biology. Assisted in selecting, coaching and properly exhibiting pieces exploring
democratisation of science and technology, speculative med-tech design, trans-humanism
and radical or guerilla bioremediation. Provided a number of workshops and residencies
on synthetic biology \& genetics.
Key skills: Science Communication, Technology Consultation, Co-Operative Design
}
%------------------------------------------------
\entry
{2007--2010}
{Cork Cancer Research Centre}
{Cork, Ireland}
{\emph{Research Assistant} \\
DNA design, cloning \& artificial gene synthesis for gene therapy vectors.
Academic review support for projects related to targeted, integrative gene therapy.
Received training in mammalian cell culture, laboratory animal welfare \& handling.
Received, processed \& stored a number of primary human tissue samples for research use.
Conducted gene therapy trials on in-vitro mammalian cell cultures.
Key skills: GMP \& HACCP, Cell culture, Microbiology, Synthetic Biology
}
%------------------------------------------------
\end{entrylist}

%----------------------------------------------------------------------------------------
%	MEMBERSHIPS SECTION
%----------------------------------------------------------------------------------------
\pagebreak
\section{Memberships}

\begin{entrylist}
%------------------------------------------------
\entry
{2014}
{Secretary}
{Cork Biomakerspace}
{\emph{(In progress)} Founding, locating and outfitting a biolab suitable for
community, outreach, educational and start-up use in Cork City Centre with support
from local industry, academia and state bodies. Areas of interest include synthetic
biology, computational biology \& bioinformatics, and medical or food technology.
}

%------------------------------------------------
\entry
{2012--2014}
{Leonardo}
{Science Gallery "Leonardo" Group, Trinity College Dublin}
{Serving as member of Leonardo think-tank, providing technological consultatoin and
creative input on present and future exhibits, knowledge and input to events, membership
on various panels and for collaborations and future exhibit curators.
Collaborations with designers, artists and technologists through the Leonardo group
have been occasional and valuable professional experiences.
}
%------------------------------------------------
\entry
{2009-2014}
{Founder}
{Nexus Cork Makerspace}
{Founded a successful maker/hacker-space (information technology club) which has
held stable residence in a central location and which has hosted a variety of
technology related events including bioinformatics workshops, 3D modelling classes
with associated 3D printing, and \href{http://corkdev.io/}{corkdev.io} programming 
meet-ups.
}
%------------------------------------------------
\entry
{2011}
{Bioethics Delegate}
{DIYbio.org Bioethics Congress}
{Participated in creating a widely accepted, terse yet expressive code of conduct
for independent researchers and amateur biotechnologists based on firm Bioethical
first principles. Disseminated and advocated this code of conduct which is now
widely accepted.
}
%------------------------------------------------
\end{entrylist}

%----------------------------------------------------------------------------------------
%	SOFTWARE SECTION
%----------------------------------------------------------------------------------------
\pagebreak
\section{Software}

\begin{entrylist}
%------------------------------------------------
\entry
{2012}
{\href{https://github.com/cathalgarvey/pysplicer}{PySplicer}}
{Github}
{Wrote a cutting edge DNA design tool, PySplicer, which implements current research
in codon and coding sequence optimisation, surpassing all other freely available tools.
Used in-house to design fluorescent protein expression constructs successfully, 
with exceptional expression levels. Contributing this to the biotechnology sector
has enabled me and others to design DNA at an accelerated pace, and has challenged a
monopoly on up-to-date DNA design tools by existing DNA synthesis providers.
}
%------------------------------------------------
\entry
{2013}
{\href{https://github.com/cathalgarvey/dncode}{DNcode}}
{Github}
{Wrote a compression function for nucleotide sequences which could rapidly \& 
losslessly compress DNA or RNA sequences to one quarter of their original storage 
size, in linear time and which could easily be implemented at database level without 
significant performance loss. The algorithm has since been re-written in several other
languages and a Mozilla Rust port is underway to provide a high-performance compiled
library for databasing applications. Given the growing surfeit of sequencing data in use and
in transit across academic networks today, DNcode could provide a much-needed reduction
in storage and bandwidth costs.
}
%------------------------------------------------
\entry
{2013}
{Litmoria}
{Private / Proprietary}
{Wrote a search-engine and data-visualisation framework for academic data, graphing
co-authorship connectivity, authorship volume, last-author frequency, \& degree-of-separation
for data from NCBI Pubmed. The engine allows arbitrary search patterns by author name
or by keyword, and has been instrumental to the client in evaluating potential academic
partners, as well as seeking unaffiliated yet academically well-placed partners for
due-diligence prior to investment or partnerships.
}
%------------------------------------------------
\entry
{2013}
{\href{https://github.com/cathalgarvey/deadlock}{deadlock}}
{Github}
{Implemented the \href{https://minilock.io}{miniLock.io} peer-reviewed cryptosystem
in Python using PyNaCl, pyblake2 and pylibscrypt. Achieved 2,000 downloads from the PyPI
repository within 48 hours of launch, currently negotiating a commercial licensing
agreement with a cloud services provider in the US for trivial, secure delivery of logging 
data to clients.
Participating in design of version 2 of the miniLock encryption scheme, with plans to
generalise encryption scheme for use in email or for securing shared folder schemes such
as Dropbox through public key cryptography.
}
%------------------------------------------------
\entry
{2013}
{\href{https://github.com/cathalgarvey/tinystatus}{tinystatus}}
{Github}
{Implemented a peer-to-peer microstatus network in 30 lines of Python (and again in
readable, properly formed Python) incorporating spam/flood resilience and ability
to "follow" not only users but arbitrary regular expressions. Feature set was limited
by the constraint that only built-in Python modules be used, yet the system was successfully
tested on a demo server for several weeks without issue.
}
%------------------------------------------------
\end{entrylist}

%----------------------------------------------------------------------------------------
%	HARDWARE SECTION
%----------------------------------------------------------------------------------------
\pagebreak

\section{Hardware}
\begin{entrylist}
%------------------------------------------------
\entry
{2009}
{\href{https://github.com/cathalgarvey/dremelfuge}{Dremelfuge}}
{Github}
{Having a need of a centrifuge for work in my own lab, and perceiving a general
need for a low-cost, high-speed microcentrifuge for medical applications in deprived
areas, I designed a 3D-printable rotor for common rotary multitools which could be
fabricated for approximately \euro1 and which could easily accomplish routine medical and
laboratory centrifugation. The designs were made available generally, and have been
exceptionally well received.
}

\entry
{2011}
{PID Water Bath Controller}
{Private Implementation}
{Needing a water-bath and scandalised by the prices quoted for such a simple piece
of equipment, I designed a microcontroller-driven feedback system and AC power controller
which could be used with domestic equipment (to wit, a kettle) to provide a water bath
solution for general use. Upgrade to an edge-immersion, fan-agitated system was partially
implemented but was found to be unnecessary.
}

\entry
{2012}
{\href{https://github.com/cathalgarvey/openpycr}{OpenPyCR}}
{Github}
{For reasons of cost-effectiveness and ease of customisation, I made use of an OpenPCR
thermal cycler for polymerase chain reaction and other enzymatic or cell reactions requiring
precise timing and temperature control. However, the software provided was poorly customisable
and required software I was unwilling to install for security reasons, so I re-implemented
the hardware control software from first principals, and extended it to allow a terse and
human-readable specification of program flow in plain text.
}

\entry
{2009}
{MicroLathe}
{Thingiverse}
{As an experiment, designed and 3D printed parts for a rotary-tool-powered miniature wood
lathe and successfully used it upon a small section of dowel. Made two revisions, with the
second providing a drastic improvment in stability while reducing printed mass. This design
was likely the first 3D printed lathe design, but has since been surpassed by more "serious"
designs.
}

\end{entrylist}

%----------------------------------------------------------------------------------------
%	BIOTECH SECTION
%----------------------------------------------------------------------------------------

\section{Biotech}
\begin{entrylist}
%------------------------------------------------
\entry
{2010}
{IndieBB (1)}
{Glowbiotics}
{Identifying a market need for a robust \& modern-standards compliant plasmid DNA vector
for \emph{B.subtilis}, I manually designed a vector which would enable antibiotic-free
selection \& maintenance and which provided "biobrick" compatible cloning sites.
The selection scheme worked as intended but the plasmid vector itself proved unstable;
I would later learn that the CDS optimisation systems I was using were just becoming
deprecated with new research, leading to the authorship of the \emph{PySplicer} DNA design
tool, above.
}

\entry
{2012-2013}
{Fluorenzymes}
{Glowbiotics}
{Desiring to produce and market a "ferment your own" enzyme product which would permit
arbitrary enzyme fermentation and purification in usefully pure form, I designed a novel
agar-based chromatographic solution and tested it with transgenic fluorescent proteins.
While proper function of PySplicer software was successfully demonstrated by vivid
fluorescence (a remarkable achievement for wild-type GFP), the agarose-binding protien
proved unamenable to protein fusion and did not function as required after several rounds
of prototyping and expert consultation.
}

\entry
{2014}
{IndieBB (2)}
{Glowbiotics}
{Designed a concept DNA vector that would enable antibiotic-free selection in \emph{E.coli}
using Colicin-V mediated allelopathic selection. Planned further experiments upon the
vector would have attempted to replace the transport cassettes of the Colicin V
operon with periplasmic disruption proteins, which would have yielded a lower DNA
footprint than extant selection cassettes.
Regrettably, funding was not raised for this project, though it is instead being implemented
as an iGEM project by a New York biotechnology lab.
}

\end{entrylist}

%----------------------------------------------------------------------------------------
%	COMMUNICATION SKILLS SECTION
%----------------------------------------------------------------------------------------

\section{Communication Events}

\begin{entrylist}
%------------------------------------------------
\entry
{2014}
{\href{http://www.pintofscience.ie}{Stem Cells, Biohacking, GM \& the future of humans}}
{Pint of Science, Dublin}
{Participated in a Panel Discussion on Biotechnology, Agriculture, Bioethics \& Human Enhancement
with positive responses from a mixed-background audience. 
}
%------------------------------------------------
\entry
{2013}
{\href{https://www.youtube.com/watch?v=g_ZswrLFSdo}{Bringing Biotechnology Into the Home}}
{TEDx Dublin 2013}
{Presented a vision for ubiquitous synthetic biology as a facet of a future everyday life at
an event attended by thousands and viewed and favoured by over ten thousand more since.
}
%------------------------------------------------
\entry
{2012}
{\href{http://vimeo.com/51144273}{Enter Bio-Hacking}}
{PICNIC Conference, Amsterdam}
{Introduced the rapidly changing field of "do-it-yourself" biotechnology at a technology
and design conference in a prominent waterfront theatre in Amsterdam.}
%------------------------------------------------
\end{entrylist}

%----------------------------------------------------------------------------------------
%	MEDIA APPEARANCES SECTION
%----------------------------------------------------------------------------------------

\section{Media Appearances}

\begin{entrylist}
%------------------------------------------------
\entry
{2012}
{\href{http://www.technologyreview.com/news/426885/doing-biotech-in-my-bedroom/}{Doing Biotech in my Bedroom}}
{MIT Technology Review}
{Coverage of a lab I established in a domestic setting and my work in enabling 
low-cost synthetic biology. (NB: There was never a bed in the lab)
}
%------------------------------------------------
\entry
{2012}
{\href{http://www.nytimes.com/2012/01/17/science/for-bio-hackers-lab-work-often-begins-at-home.html?_r=0}{For Bio-hackers, Lab Work Often Begins at Home}}
{New York Times}
{Headed a story on DIY-biotechnology with reference to my work designing low-cost tools and methods.
}
%------------------------------------------------
\entry
{2013}
{Biohackers - Les Bricoleurs d'ADN}
{Le Monde (Online)}
{Presented alongside other leading figures in Lean Biotech in an interactive online
format which encouraged viewers to "ask" questions and receive pre-recorded responses.
Initially available to the public, but since paywalled. 
}
%------------------------------------------------

\end{entrylist}


\end{document}
